\documentclass[11pt]{jsarticle}

\usepackage[top=30mm, bottom=36mm, left=28mm, right=28mm]{geometry}
\usepackage{url}

\usepackage{mytitle}

\title{0. チュートリアル概要}
\author{201720690 小松 弘人}
\date{2017/06/12}

\makeatletter
\def\mojiparline#1{
    \newcounter{mpl}
    \setcounter{mpl}{#1}
    \@tempdima=\linewidth
    \advance\@tempdima by-\value{mpl}zw
    \addtocounter{mpl}{-1}
    \divide\@tempdima by \value{mpl}
    \advance\kanjiskip by\@tempdima
    \advance\parindent by\@tempdima
}
\makeatother
\def\linesparpage#1{
    \baselineskip=\textheight
    \divide\baselineskip by #1
}

\begin{document}
\maketitle
\subsection*{演習概要}
FPGAによる処理の基本および開発方法の基礎を学ぶため、
画像のストリーム処理についての簡単な演習を行う。
画像フィルタは、計算の簡単さから$3\!\times\!3$
ガウシアンフィルタを実装することとする。
本チュートリアルでは、プラットフォームとして
ZedBoard上でXillinuxを起動し、その上で演習を行っていく。

\subsection*{演習で学ぶこと}
\begin{itemize}
	\item
		ストリーム処理の基礎
	\item
		Vivadoでの開発方法の基礎
	\item
		Verilogの基礎
\end{itemize}

\subsection*{演習期間の目安}
1週間~

\subsection*{事前に準備するもの}
\begin{itemize}
	\item
		ZedBoardとその周辺機器 (VGAディスプレイ、キーボード、マウス、USBハブなど)
	\item
		$2\mathrm{[GByte]}$以上のSDカード
	\item
		Vivado 2014.4以降 (ライセンスはWebPACKでOK)
	\item
		演習に必要なファイル一式
\end{itemize}

\subsection*{演習に必要なファイル}
GitHub (\url{https://github.com/HDL-tutorial})に、
演習に必要なデータをアップロードしている。
それぞれのリポジトリには、次のようなデータが入っている。

\begin{itemize}
	\item
		documents: このドキュメントを含む、本演習のドキュメント集
	\item
		xillinux: Vivadoプロジェクトなどを含むXillinuxのブートパーティションキット
	\item
		software: ZedBoard上で動作させるC++プログラム
	\item
		out-sim-software: 実機で得られた結果画像が正しいか確認するためのプログラム
\end{itemize}

各リポジトリのページから、ZIPでダウンロード、もしくはgitを使ってクローンすること。

\end{document}
