\documentclass[11pt]{jsarticle}

\usepackage[top=30mm, bottom=36mm, left=28mm, right=28mm]{geometry}

\usepackage{mytitle}

\title{画像のループバック演習}
\author{201720690 小松 弘人}
\date{2017/06/12}

\makeatletter
\def\mojiparline#1{
    \newcounter{mpl}
    \setcounter{mpl}{#1}
    \@tempdima=\linewidth
    \advance\@tempdima by-\value{mpl}zw
    \addtocounter{mpl}{-1}
    \divide\@tempdima by \value{mpl}
    \advance\kanjiskip by\@tempdima
    \advance\parindent by\@tempdima
}
\makeatother
\def\linesparpage#1{
    \baselineskip=\textheight
    \divide\baselineskip by #1
}

\begin{document}
\maketitle
\subsection*{演習目的}
本演習の進め方を理解する。
FIFOの基礎を実践しながら理解する。

\subsection*{演習概要}
ZedBoardでXillinuxを起動し、その上でARMからFPGAに画像データを送るプログラムを
実行し、結果画像を確認する。
そのために、まずはVivadoで画像をループバックさせるための回路を記述する。
次に、ブートに必要なファイルをSDカードに入れ、ZedBoard上でXillinuxを起動する。
最後に、結果が正しいかどうかを確認する。

\subsection*{演習内容}
\subsubsection*{Vivadoでxillydemoプロジェクトを作成する}
\subsubsection*{Xillinuxのブート用SDカードを作成する}
\subsubsection*{Xillinuxの環境設定・事前準備}
\subsubsection*{演習用プログラムの実行}
\subsubsection*{結果画像の確認}

\end{document}
