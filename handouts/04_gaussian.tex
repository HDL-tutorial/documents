\documentclass[11pt]{jsarticle}

\usepackage[top=30mm, bottom=36mm, left=28mm, right=28mm]{geometry}

\usepackage{mytitle}

\title{4. Gaussian Filterの作成}
\author{201720690 小松 弘人}
\date{2017/06/12}

\makeatletter
\def\mojiparline#1{
    \newcounter{mpl}
    \setcounter{mpl}{#1}
    \@tempdima=\linewidth
    \advance\@tempdima by-\value{mpl}zw
    \addtocounter{mpl}{-1}
    \divide\@tempdima by \value{mpl}
    \advance\kanjiskip by\@tempdima
    \advance\parindent by\@tempdima
}
\makeatother
\def\linesparpage#1{
    \baselineskip=\textheight
    \divide\baselineskip by #1
}

\begin{document}
\maketitle
\subsection*{演習目的}
簡単なストリーム処理回路の設計およびシミュレーション方法を学ぶ。

\subsection*{演習概要}
$3\!\times\!3$ガウシアンフィルタを実装し、実機で確認する。

\subsection*{演習内容}
\subsubsection*{事前準備}
``3. 画像のループバック''の演習を終わらせておくこと。

\subsubsection*{gaussianモジュールの作成 (Windows)}
前回同様に、\verb|gaussian.v|を作成する。
ただし、入力は\verb|duo|, \verb|dco|, \verb|dlo|の3つである。
それぞれが、上段、中段、下段の入力である。
それぞれの段の入力信号の下位$8\mathrm{[bit]}$は、
RGBをY (輝度) に変換した値である。
したがって、\verb|gaussian.v|の\verb|duo|, \verb|dco|, \verb|dlo|は
それぞれ$8\mathrm{[bit]}$でよい。
\verb|gaussian.v|では、これらの入力に対しGaussian Filter
(式\ref{eq:gaussian}) を計算し、計算結果を
出力側のFWFTモードのFIFOに格納する回路を作成する。
このFIFOのデータ幅は$32\mathrm{[bit]}$である。
そこで、輝度を3回繰り返し、先頭を0埋めしたものをFIFOに入力する。

\begin{equation}
	\label{eq:gaussian}
	H(x, y)\!=\!\frac{1}{16}\!\left(
	\begin{array}{ccc}
		1 & 2 & 1 \\
		2 & 4 & 2 \\
		1 & 2 & 1 \\
	\end{array}
	\right)
\end{equation}

\subsubsection*{回路のシミュレーション (Windows)}
Vivadoシミュレータを使用する。
xillinuxリポジトリの中には、シミュレーション用のプロジェクトを
作成するためのtclスクリプトが含まれている。
これを前回と同様に実行する。
Vivadoを起動し、Tools→Run Tcl Script...から
userfiles/sim/create\_project.tclを実行する。
しばらく待つと、wrapperプロジェクトが作成され、
プロジェクトの内容が表示される。

Vivadoシミュレータは、Flow NavigatorからRun Simulationをクリックし、
Run Behavioral Simulationを押すことで起動できる。
wrapperプロジェクトでは、wrapper.vに対し、画素値が左上から
RGB=\verb|0x010101|, \verb|0x020202|, ...となっている$8\!\times\!8$
の画像データを送るようなテストベンチが用意されている。
Vivadoシミュレータが起動すると、シミュレーション結果が表示される。
値を確認したい信号がその中に含まれていない場合、
その信号をScopeから選択し、RestartしてからRunすることで、
その信号の値を確認できる。

\subsubsection*{演習用プログラムの実行 (Xillinux)}
前回と同様に\verb|bin/send_bmp|を使用する。
プログラムが終了しない場合は、シミュレータで
\verb|wr_en|や\verb|rd_en|の値が正しいか確認すること。

\subsubsection*{結果画像の確認 (Windows)}
チュートリアルのページから、out-sim-softwareリポジトリを
ダウンロードしておくこと。
これは、Visual Studioのプロジェクトである。
\verb|main.cpp|を編集し、画像のパス (11行目 \verb|SRC_FILE|)
を変更する。
このプログラムは、3つの画像を出力する。
これらの3つの画像は、それぞれ以下のような画像である。
3つの画像の違いは、範囲外の画素値の処理のみである。
\begin{itemize}
	\item
		dst0.bmp: Vivadoのシミュレータの1枚目の出力結果
	\item
		dst1.bmp: Vivadoのシミュレータの2枚目以降の出力結果
	\item
		dst2.bmp: 実機で処理した結果として出力される画像
\end{itemize}

\end{document}
