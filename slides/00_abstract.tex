\documentclass[dvipdfmx]{beamer}
\usepackage{pxjahyper}
\usepackage{bookmark}
\usepackage{txfonts}

\usetheme{Copenhagen}
\usecolortheme{rose}

\renewcommand{\familydefault}{\sfdefault}
\renewcommand{\kanjifamilydefault}{\gtdefault}

\setbeamercovered{dynamic}

\setbeamertemplate{headline}{}
\setbeamertemplate{footline}[page number]
\setbeamertemplate{navigation symbols}{}
\setbeamertemplate{section in toc}[sections numbered]
\setbeamertemplate{items}[default]
\setbeamertemplate{blocks}[rounded]

\usefonttheme{professionalfonts}
\usefonttheme[onlymath]{serif}
\usefonttheme{structurebold}
\setbeamerfont{frametitle}{size=\Large}
\setbeamerfont{title}{size=\Large}

\title{0. チュートリアル概要}
\author{201720690 小松 弘人}
\date{\today}

\begin{document}

\maketitle

\begin{frame}
	\frametitle{Agenda}
	\tableofcontents
\end{frame}

\section{演習の目的}
\begin{frame}
	\frametitle{演習の目的}

	\begin{itemize}
		\item
			FPGAによる処理の基本を学習
			\vfill
		\item 
			開発方法の基礎を学習
			\vfill
		\item
			ストリーム処理を実践
			\vfill
		\item
			Verilogに親しむ
	\end{itemize}
\end{frame}

\section{演習の概要}
\begin{frame}
	\frametitle{演習の概要}
	この演習は、画像のストリーム処理についての簡単な演習である
	\begin{enumerate}
		\item
			FIFO IP Coreについて学ぶ
			\vfill
		\item
			ストリーム処理の基礎を確認
			\vfill
		\item
			$3\!\times\!3$ガウシアンフィルタを設計・実装
			\vfill
		\item
			ZedBoard上でXillinuxを起動し、動作を確認
	\end{enumerate}
\end{frame}

\section{演習にかかる目安}
\begin{frame}
	\frametitle{演習にかかる目安}
	\begin{itemize}
		\item
			1週間~
	\end{itemize}
	\begin{description}
		\item[説明]\mbox{}\\
			1日程度
		\item[実装]\mbox{}\\
			1週間~
	\end{description}
\end{frame}

\section{事前準備}
\begin{frame}
	\frametitle{事前準備}
	\begin{itemize}
		\item
			ZedBoardとその周辺機器
			\begin{itemize}
				\item
					VGAディスプレイ
				\item
					キーボード
				\item
					マウス
				\item
					USBハブなど
			\end{itemize}
			\vfill
		\item
			$2\mathrm{[GByte]}$以上のSDカード
			\vfill
		\item
			Vivado 2014.4以降 (ライセンスはWebPACKでOK)
			\vfill
		\item
			Microsoft Visual Studio 2015以降
			\vfill
		\item
			演習に必要なファイル一式 (GitHubからダウンロード)
			\vfill
	\end{itemize}
\end{frame}

\section{演習に必要なファイル}
\begin{frame}
	\frametitle{演習に必要なファイル}
	
	GitHub (\url{https://github.com/HDL-tutorial})に、\\
	演習に必要なデータをアップロードしている
	\vfill
	\begin{description}
		\item[documents]\mbox{}
			本演習のドキュメント集
			\vfill
		\item[xillinux]\mbox{}
			Xillinuxのブートパーティションキット
			\vfill
		\item[software]\mbox{}
			ZedBoard上で動作させるC++プログラム
			\vfill
		\item[out-sim-software]\mbox{}
			結果を確認するためのC++プログラム
			\vfill
	\end{description}
\end{frame}

\end{document}
