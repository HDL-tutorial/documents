\documentclass[dvipdfmx]{beamer}
\usepackage{pxjahyper}
\usepackage{bookmark}
\usepackage{txfonts}

\usetheme{Copenhagen}
\usecolortheme{rose}

\renewcommand{\familydefault}{\sfdefault}
\renewcommand{\kanjifamilydefault}{\gtdefault}

\renewcommand{\figurename}{図}
\renewcommand{\tablename}{表}

\setbeamercovered{dynamic}

\setbeamertemplate{headline}{}
\setbeamertemplate{footline}[page number]
\setbeamertemplate{navigation symbols}{}
\setbeamertemplate{section in toc}[sections numbered]
\setbeamertemplate{items}[default]
\setbeamertemplate{blocks}[rounded]

\usefonttheme{professionalfonts}
\usefonttheme[onlymath]{serif}
\usefonttheme{structurebold}
\setbeamerfont{frametitle}{size=\Large}
\setbeamerfont{title}{size=\Large}

\title{4. Gaussian Filterの作成}
\author{201720690 小松 弘人}
\date{\today}

\begin{document}

\maketitle

\begin{frame}
	\frametitle{Agenda}
	\tableofcontents
\end{frame}

\section{演習の目的}
\begin{frame}
	\frametitle{演習の目的}
	\begin{itemize}
		\item
			簡単なストリーム処理回路の設計およびシミュレーション方法を学ぶ
	\end{itemize}
\end{frame}

\section{演習の概要}
\begin{frame}
	\frametitle{演習の概要}
	\begin{itemize}
		\item
			$3\!\times\!3$ガウシアンフィルタを実装し、実機で確認する
	\end{itemize}
\end{frame}

\section{gaussianモジュールの作成 (Windows)}
\begin{frame}
	\frametitle{gaussianモジュールの作成 (Windows)}
	\begin{enumerate}
		\item
			gaussian.vを作成する\\
			\begin{itemize}
				\item
					データ入力信号をdui, dci, dliに増やす
				\item
					それぞれの入力の下位$8\mathrm{[bit]}$に輝度が入っている
			\end{itemize}
		\item
			wrapper.vを編集
		\item
			gaussian.vを編集 (式の係数を用いる)
			\begin{itemize}
				\item
					出力は輝度を3回繰り返し、先頭を0埋めしたものにする
			\end{itemize}
	\end{enumerate}
	\begin{equation}
		\label{eq:gaussian}
		H(x, y)\!=\!\frac{1}{16}\!\left(
		\begin{array}{ccc}
			1 & 2 & 1 \\
			2 & 4 & 2 \\
			1 & 2 & 1 \\
		\end{array}
		\right)
	\end{equation}
\end{frame}

\section{回路のシミュレーション (Windows)}
\begin{frame}
	\frametitle{回路のシミュレーション (Windows)}
	\begin{enumerate}
		\item
			Vivadoシミュレータを使用する
		\item
			Vivadoを起動
		\item
			Tools→Run Tcl Script...で
			userfiles/sim/create\_project.tclを実行
		\item
			wrapperプロジェクトが開くまで待つ
		\item
			Flow NavigatorからRun Simulation→Run Behavioral Simulation
			をクリックし、Vivadoシミュレータを起動
		\item
			起動すると、シミュレーション結果が表示される
		\item
			確認したい信号が結果に含まれていない場合、
			Scopeで信号を探し、波形表示の上にドラッグ\&ドロップし、
			Restart\&Runで追加できる
	\end{enumerate}
\end{frame}

\begin{frame}[fragile]
	\frametitle{wrapperプロジェクト}
	\begin{itemize}
		\item
			以下のようなテストベンチを用意
			\begin{itemize}
				\item
					wrapper.vに対してデータを入力
					\vfill
				\item
					画素値が左上から順に
					\verb|0x010101|, \verb|0x020202|, ...
					となっている$8\!\times\!8$の画像データを送る
					\vfill
			\end{itemize}
	\end{itemize}
\end{frame}

\section{演習用プログラムの実行 (Xillinux)}
\begin{frame}[fragile]
	\frametitle{演習用プログラムの実行 (Xillinux)}
	\begin{itemize}
		\item
			前回と同様に\verb|bin/send_bmp|を使用する。
			\vfill
		\item
			プログラムが終了しない場合は、シミュレータで
			\verb|wr_en|や\verb|rd_en|の値が正しいか確認すること。
	\end{itemize}
\end{frame}

\section{出力画像の確認 (Windows)}
\begin{frame}[fragile]
	\frametitle{出力画像の確認 (Windows)}
	\begin{itemize}
		\item
			out-sim-softwareリポジトリをダウンロードする
			\begin{itemize}
				\item
					Visual Studioのプロジェクト
			\end{itemize}
			\vfill
		\item
			\verb|main.cpp|を編集し、
			画像のパス (11行目 \verb|SRC_FILE|)を変更
			\vfill
		\item
			3種類の出力画像を生成する
			\begin{itemize}
				\item
					違いは範囲外の画素値の処理のみ
					\begin{description}
						\item[dst0.bmp]\mbox{}
							Vivadoのシミュレータの1枚目の出力
						\item[dst1.bmp]\mbox{}
							Vivadoのシミュレータの2枚目以降の出力
						\item[dst2.bmp]\mbox{}
							実機で処理した結果として出力される画像
					\end{description}
			\end{itemize}
	\end{itemize}
\end{frame}

\end{document}
